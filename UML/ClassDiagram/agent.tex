\documentclass[11pt]{article}
\usepackage[T1]{fontenc}
\usepackage[polish]{babel}
\usepackage[utf8]{inputenc}


\begin{document}
\section{Agent}
\subsection{Zmienne}
\begin{center}
    \begin{tabular}{ | l | l | l | p{5cm} |}
    \hline
    Dostęp & Typ & Nazwa & Opis \\ \hline
    private & int & id & Identyfikator gracza \\ \hline
    public & int & delayTime & Czas między wykonywanymi akcjami \\ \hline
    public & int & penaltyTime & Czas kary \\ \hline
    public & Team & team & Enum drużyny, w której jest gracz \\ \hline
    public & bool & isLeader & Zmienna informująca czy gracz jest liderem zespołu \\ \hline
    public & bool & havePiece & Zmienna informująca czy gracz jest w posiadaniu piece'a \\ \hline
    public & Field[,] & board & Tablica dwuwymiarowa przechowująca planszę z perspektywy gracza \\ \hline
    public & Tuple<int,int> & position & Współrzędne pola na planszy, na którym stoi gracz \\ \hline
    public & List<Tuple<int,bool>> & position & Lista id oraz 
    informacji czy są oni liderami drużyny graczy oczekujących na odpowiedź \\ \hline
    public & int[] & teamMates & Lista identyfikatorów graczy z naszej drużyny \\ \hline
    private & IStrategy & strategy & Obiekt odpowiedzialny za używaną strategię \\ \hline
    \end{tabular}
\end{center}

\subsection{Metody}
\begin{center}
    \begin{tabular}{ | l | l | l | p{5cm} |}
    \hline
    Dostęp & Zwracany typ & Nazwa & Opis \\ \hline
    public & void & Player() & Konstruktor klasy \\ \hline
    public & void & JoinTheGame() & Metoda dołączająca gracza do gry \\ \hline
    public & void & Start() & Metoda rozpoczynająca grę \\ \hline
    public & void & Stop() & Metoda zatrzymująca grę \\ \hline
    public & void & Move() & Metoda ruszająca gracza na wskazane pole \\ \hline
    public & void & PickUp() & Meteda podnosząca z planszy fragment \\ \hline
    public & void & Put() & Metoda odkładająca fragment na planszę \\ \hline
    public & void & BegForInfo() & Metoda wysyłająca prośbę o udzielenie informacji \\ \hline
    public & void & GiveInfo() & Metoda udzielająca informacji o rozgrywce wskazanemu graczowi \\ \hline
    public & void & RequestsResponse() & Metoda odpowiadająca na zapytania innych graczy \\ \hline
    public & void & CheckPiece() & Metoda sprawdzająca czy fragment to fragment fikcyjny \\ \hline
    public & void & GetInfoFromGM() & Metoda pobierająca informacje od communication serwera \\ \hline
    public & void & MakeDecisionFromStrategy() & Metoda, która wywołuje metodę z obiektu strategy \\ \hline
    private & void & Communicate() & Metoda wysyłająca wiadomości do communication serwera \\ \hline
    private & void & Penalty() & Funkcja czekająca przez okres kary \\ \hline
    \end{tabular}
\end{center}

\section{Interface IStratety}
\subsection{Metody}
\begin{center}
\begin{tabular}{ | l | l | l | p{5cm} |}
    \hline
    Dostęp & Zwracany typ & Nazwa & Opis \\ \hline
    public & void & MakeDecision & Metoda, która decyduje jaką akcją wykona gracz \\ \hline
    \end{tabular}
\end{center}
\section{Klasa Field}
\subsection{Zmienne}
\begin{center}
    \begin{tabular}{ | l | l | l | p{5cm} |}
    \hline
    Dostęp & Typ & Nazwa & Opis \\ \hline
    public & GoalInfo & goalInfo & Stan wiedzy gracza na temat tego pola \\ \hline
    public & bool & playerInfo & Informacja czy inny gracz stoi na tym polu \\ \hline
    public & int & distToPiece & Odległość pola od najbliższego fragmentu \\ \hline
    \end{tabular}
\end{center}
\section{Enum Team}
\begin{itemize}
    \item \textbf{Red} - informacja, że gracz jest w drużynie czerwonej,
    \item \textbf{Blue} - informacja, że gracz jest w drużynie niebieskiej.
\end{itemize}
\section{Enum GoalInfo}
\begin{itemize}
    \item \textbf{IDK} - gracz nic nie wie o danym polu,
    \item \textbf{DiscoveredNotGoal} - pole nie jest celem,
    \item \textbf{DiscoveredGoal} - pole jest celem i jest na nim położony fragment,
\end{itemize}
\end{document}