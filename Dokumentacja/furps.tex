\documentclass[Dokumentacja.tex]{subfiles}

\begin{document}
\section{FURPS - wymagania niefunkcjonalne}
\subsection{Usability}
\begin{itemize}
    \item system powinien działać po podaniu właściwych opcji konfiguracyjnych,
    \item moduł GM powinien wyświetlać estetyczną planszę z przebiegiem rozgrywki,
    \item system powinien działać na domyślnej konfiguracji modułów.
\end{itemize}

\subsection{Reliability}
\begin{itemize}
    \item system powinien być niewrażliwy na ewentualne odłączenie się innych modułów,
    wykryć takie zachowanie i odpowiednio zareagować,
    \item system powinien być odporny na przesyłanie do poszczególnych modułów niepoprawnych wiadomości,
    \item serwer komunikacyjny powinien zapisywać logi z niepoprawnymi wiadomościami.
\end{itemize}

\subsection{Performance}
\begin{itemize}
    \item Instalując moduł Game Mastera i serwera komunikacyjnego na przeciętnym laptopie, system powinien obsłużyć około 40 agentów,
    a wielkość planszy powinna mieć pomijalny wpływ na wydajność gry,
    \item skalowalność systemu powinna zależeć od ilości wątków procesora ze względu na asynchroniczny charakter pracy modułów
    Game Mastera i serwera komunikacyjnego.
\end{itemize}

\subsection{Supportability}
\begin{itemize}
    \item system powinien mieć domyślne ustawienia, które zostają każdorazowo nadpisane przez odpowiednie pliki konfiguracyjne,
    \item system powinien działać w sieciach wspierających protokół TCP/IP,
    \item system powinien być aplikacją typu portable, czyli taką którą można uruchomić bez instalacji,
    \item system powinien być napisany możliwe przenośnie, to znaczy maksymalnie ułatwiając jego przeniesienie na inny system operacyjny.
\end{itemize}

\end{document}
