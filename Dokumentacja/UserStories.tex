\documentclass[Dokumentacja.tex]{subfiles}

\begin{document}

\section{User stories}
\subsection{Użytkownik}
\subsubsection{W module serwera}
\begin{itemize}
    \item Chce ustawić parametry serwera ?` port jeśli TCP ?
\end{itemize}

\subsubsection{W module GM}
\begin{itemize}
    \item Chce skonfigurować połączenie z modułem serwera
    \item Chce skonfigurować parametry gry
    \begin{itemize}
        \item wielkość planszy
        \item wielkość pola bramkowego
        \item ilość celów ?` i jaki kształt stanowią ?
        \item opóźnienia między poszczególnymi czynnościami agentów
        \item ilość agentów
        \item prawdopodobieństwo wygenerowanie fragmentu fikcyjnego
        \item częstotliwość generowania nowych celów oraz maksymalna ilość celów na planszy (wliczając obecnie podniesione fragmenty)
    \end{itemize}
    \item Chce wyświetlić planszę
    \item Chce rozpocząć rozgrywkę
    \item Chce zakończyć rozgrywkę
\end{itemize}


\subsubsection{W module agenta}
\begin{itemize}
    \item Chce skonfigurować połączenie z modułem serwera
    \item Chce ustawić strategię
    \item Chce ustawić drużynę do której należy
\end{itemize}


\subsection{Agent}
\begin{itemize}
    \item Chce dołączyć do rozgrywki/drużyny
    \item Chce wybrać kolejną akcję na podstawie obranej strategii
    \item Chce odpytać o poprawność wykonania i wykonać akcję:
    \begin{itemize}
        \item ruch
        \item akcję odkrycia
        \item odłożenie fragmentu
        \item sprawdzenie czy fragment jest fikcyjny
        \item zapytanie o wiedzę innego agenta
    \end{itemize}
    \item Chce odbierać wiadomości i poprawnie reagować na nie
    \begin{itemize}
        \item Walidacja akcji; zmieniając stan lokalnej planszy
        \item Zakończenie rozgrywki; kończąc pracę
        \item Zapytanie o wiedzę; odsyłając posiadane informacje
    \end{itemize}
\end{itemize}

\subsection{GM}
\begin{itemize}
    \item Chce walidować poprawność otrzymanego zapytania
    \item Chce aktualizować stan planszy w przypadku zapytań modyfikujących jej
    \item Chce odpowiadać na zapytania
    \item Chce rozesłać informację o zakończeniu rozgrywki
    \item Chce obliczyć podsumowanie rozgrywki
\end{itemize}
\end{document}
