\documentclass[Dokumentacja.tex]{subfiles}

\begin{document}
\section{Reguły}
\subsection{Założenia ogólne}
\begin{itemize}
    \item Powierzchnia pól bramkowych jest równa dla obu drużyn
    \item Pole bramkowe zawsze ma szerokość taką jak szerokość pola planszy
    \item Pola bramkowe stanowi $n$ pierwszych i $n$ ostatnich wierszy planszy, $n > 0$
    \item Pola bramkowe są spójne
    \item Powierzchnia pola zadań oraz powierzchnia obu pól bramkowych są niezerowe
    \item Pola bramkowe (ułożenie celów) obu drużyn są symetryczne względem środka planszy
    \item Odległości liczone są w metryce Manhattan
    \item Każda akcja powoduje założenie na agenta blokady na czas zależny od akcji i określony przed rozgrywką podczas której nie może wykonywać akcji
    \item W każdej drużynie jest jeden lider
\end{itemize}

\subsection{Poruszanie}
\begin{itemize}
    \item Agent może poruszać się w 4 kierunkach: zachód, północ, wschód, południe, to jest; może przemieścić się na pole które jest przyległe do pola na którym się znajduje
    \item Agent może poruszyć się na dowolne pole planszy
    \item Próba ruchu poza planszę skutkuje brakiem ruchu
    \item Próba poruszenia się na pole zajęte przez innego agenta skutkuje brakiem ruchu
    \item Próba poruszenia się dwóch różnych agentów na te samo pole skutkuje poruszeniem jednego z nich, wybór agenta do poruszenia jest dowolny
\end{itemize}
\subsection{Fragmenty}
\begin{itemize}
    \item Na Każdym polu planszy, które nie jest polem bramkowym może znajdować się dowolna ilość fragmentów
    \item Agent może trzymać maksymalnie jeden fragment
    \item W każdym momencie w grze może znajdować się maksymalnie liczba fragmentów, która ustalana jest przed rozpoczęciem rozgrywki w konfiguracji
    \item Fragmenty generowane są przez Game Mastera co określony interwał i odkładane na losowe pole planszy
    \item Odłożenie wygenerowanego przez Game Mastera fragmentu na pole na którym znajduje się Agent skutkuje natychmiastowym podniesieniem fragmentu przez tego Agenta, jeśli nie trzyma on już innego fragmentu, w przeciwnym przypadku fragment odkładany jest na pole
    \item Odłożenie fragmentu niefikcyjnego na polu bramkowym które jest celem skutkuje przyznaniem punktu drużynie do której należy dane pole bramkowe. Może nastąpić sytuacja, gdy Agent zdobywa punkt dla przeciwnej drużyny
    \item Odłożenie fragmentu fikcyjnego na polu bramkowym zwraca informację agentowi, że dany fragment był fikcyjny oraz nie przyznaje punktów, nawet jeśli dane pole było celem
    \item Cel na który został odłożony fragment niefikcyjny przestaje być celem
    \item Odłożenie fragmentu na polu bramkowym skutkuje wyłączeniem tego fragmentu z gry
    \item Odłożenie fragmentu na polu niebędącym polem bramkowym jest niedozwolone
    \item Fragment trzymany przez gracza w każdym momencie może być przetestowany na bycie fikcyjnym
    \item Fragment trzymany przez gracza w każdym momencie może być zniszczony
\end{itemize}
\subsection{Wymiana informacji}
\begin{itemize}
    \item W każdym momencie gry agent może zarządzać wymiany informacji z innym agentem
    \item Odpowiedź na zapytanie agenta niebędącego liderem nie jest obligatoryjna
    \item Odpowiedź na zapytanie agenta będącego liderem jest obligatoryjna i skutkuje zablokowaniem akcji pytanego agenta do udzielenia przez niego odpowiedzi
    \item Odpowiedź na zapytanie może zostać udzielona w dowolnym czasie od zapytania o wymianę informacji, ale nie może zostać wysłana jeśli takie zapytanie nie miało wcześniej miejsca
    \item Udzielane informacje nie muszą być prawdziwe
\end{itemize}

\end{document}
