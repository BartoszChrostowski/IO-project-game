\documentclass[Dokumentacja.tex]{subfiles}

\begin{document}
\section{Komunikacja}
\label{sec:komunikacja}

Komunikacja odbywa się za pośrednictwem protokołu TCP. Pierwsze 2 bajty zapisane w kolejności \texttt{little endian}
Oznaczają liczbę bajtów wiadomości do odczytania (wyłączając pierwsze dwa bajty).
Wiadomości nie powinny przekraczać długości 8 KiB. Wiadomości kodowane są w UTF-8.
Utrata połączenia z którymkolwiek modułem skutkuje w każdym z modułów
poinformowaniem użytkownika i zakończeniem działania. Wiadomości wysyłane są w formacie JSON.
Wiadomości wysyłane przez GM i Agentów opakowywane są w adapter opisujący kontekst wysyłanej wiadomości.
Pole \textit{agentID} uzupełniane jest przez serwer na podstawie jego własnego mapowania
klientów na identyfikatory. Po nawiązaniu połączenia przez Agenta z serwerem,
serwer nadaje mu unikane \textit{agentID} (liczbę całkowitą), które będzie służyło do korelacji oraz będzie
wykorzystywane przez GM w logice gry.
\textit{agentID} może być unikalną losowa wartością lub kolejną wartością predefiniowanego ciągu
unikalnych wartości. To samo \textit{agentID}
wysyłane jest do Agenta w odpowiedzi na prośbę o dołączenie
oraz w wiadomości o rozpoczęciu rozgrywki. Ponowne połączenia po utracie komunikacji nie są możliwe.

\lstinputlisting[language=json]{./DefinicjeWiadomosci/messageWrapper.json}
\subsection{Wiadomości Agenta}
\subsubsection{Zapytanie czy trzymany fragment jest fikcyjny}
\textit{Payload} pusty
\lstinputlisting[language=json]{./DefinicjeWiadomosci/AgentMes/checkForScham.json}

\subsubsection{Zapytanie o zniszczenie fragmentu}
\textit{Payload} pusty
\lstinputlisting[language=json]{./DefinicjeWiadomosci/AgentMes/destroyPiece.json}

\subsubsection{Zapytanie o akcję odkrycia}
\textit{Payload} pusty
\lstinputlisting[language=json]{./DefinicjeWiadomosci/AgentMes/discovery.json}

\subsubsection{Odpowiedź na wymianę informacji}
\lstinputlisting[language=json]{./DefinicjeWiadomosci/AgentMes/infomationExchangeRespond.json}

\subsubsection{Zapytanie o wymianę informacji}
\lstinputlisting[language=json]{./DefinicjeWiadomosci/AgentMes/informationExchangeAsk.json}

\subsubsection{Zapytanie o dołączenie do rozgrywki}
\lstinputlisting[language=json]{./DefinicjeWiadomosci/AgentMes/joinGame.json}

\subsubsection{Zapytanie o ruch}
\lstinputlisting[language=json]{./DefinicjeWiadomosci/AgentMes/makeMove.json}

\subsubsection{Zapytanie o podniesienie fragmentu}
\textit{Payload} pusty
\lstinputlisting[language=json]{./DefinicjeWiadomosci/AgentMes/pickPiece.json}

\subsubsection{Zapytanie o położenie fragmentu}
\textit{Payload} pusty
\lstinputlisting[language=json]{./DefinicjeWiadomosci/AgentMes/putPiece.json}


\subsection{Wiadomości GM}
\subsubsection{Odpowiedź na sprawdzenie fikcyjności}
\lstinputlisting[language=json]{./DefinicjeWiadomosci/GMMes/checkForSchamResponse.json}

\subsubsection{Odpowiedź na prośbę o zniszczenie fragmentu}
\lstinputlisting[language=json]{./DefinicjeWiadomosci/GMMes/destroyPieceResponse.json}

\subsubsection{Odpowiedź na akcję discovery}
\lstinputlisting[language=json]{./DefinicjeWiadomosci/GMMes/discoveryResponse.json}

\subsubsection{Wiadomość o zakończeniu gry}
\lstinputlisting[language=json]{./DefinicjeWiadomosci/GMMes/gameEnd.json}

\subsubsection{Wiadomość o rozpoczęciu gry}
\lstinputlisting[language=json]{./DefinicjeWiadomosci/GMMes/gameStart.json}

\subsubsection{Wiadomość przekazująca zapytanie o wymianę informacji do adresata}
\lstinputlisting[language=json]{./DefinicjeWiadomosci/GMMes/informationExchangeAskForward.json}

\subsubsection{Odpowiedź na zapytanie o dołączenie}
\lstinputlisting[language=json]{./DefinicjeWiadomosci/GMMes/joinGameResponse.json}

\subsubsection{Odpowiedź na zapytanie o ruch}
\lstinputlisting[language=json]{./DefinicjeWiadomosci/GMMes/makeMoveResponse.json}

\subsubsection{Odpowiedź na podniesienie kawałka}
\lstinputlisting[language=json]{./DefinicjeWiadomosci/GMMes/pickPieceResponse.json}

\subsubsection{Odpowiedź na położenie kawałka}
\lstinputlisting[language=json]{./DefinicjeWiadomosci/GMMes/putPieceResponse.json}

\subsubsection{Wiadomość przekazująca odpowiedź na zapytanie o wymianę informacji}
\lstinputlisting[language=json]{./DefinicjeWiadomosci/GMMes/informationExchangeRespondForward.json}

\subsection{Wiadomości błędów}

\subsubsection{Błędny ruch}
\lstinputlisting[language=json]{./DefinicjeWiadomosci/Errors/invalidMove.json}

\subsubsection{Błąd podniesienia kawałka}
\lstinputlisting[language=json]{./DefinicjeWiadomosci/Errors/pickPiece.json}

\subsubsection{Błąd położenia kawałka}
\lstinputlisting[language=json]{./DefinicjeWiadomosci/Errors/putPieceError.json}

\subsubsection{Nie odczekanie kary}
\lstinputlisting[language=json]{./DefinicjeWiadomosci/Errors/timePenaltyError.json}

\subsubsection{Niezdefiniowany błąd}
\lstinputlisting[language=json]{./DefinicjeWiadomosci/Errors/undefinedErrorMessage.json}

\subsection{Id wiadomości}

\subsubsection{Wiadomości Agenta}
\begin{tabular}{ |p{1cm}|p{12cm}| }
 \hline
 Id & Nazwa wiadomości \\
 \hline
 001 & Zapytanie czy trzymany fragment jest fikcyjny \\
 002 & Zapytanie o zniszczenie fragmentu \\
 003 & Zapytanie o akcję odkrycia \\
 004 & Odpowiedź na wymianę informacji \\
 005 & Zapytanie o wymianę informacji \\
 006 & Zapytanie o dołączenie do rozgrywki \\
 007 & Zapytanie o ruch \\
 008 & Zapytanie o podniesienie fragmentu \\
 009 & Zapytanie o położenie fragmentu \\
 \hline
\end{tabular}

\subsubsection{Wiadomości GM}
\begin{tabular}{ |p{1cm}|p{12cm}| }
 \hline
 Id & Nazwa wiadomości \\
 \hline
 101 & Odpowiedź na sprawdzenie fikcyjności \\
 102 & Odpowiedź na prośbę o zniszczenie fragmentu \\
 103 & Odpowiedź na akcję discovery \\
 104 & Wiadomość o zakończeniu gry \\
 105 & Wiadomość o rozpoczęciu gry \\
 106 & Wiadomość przekazująca zapytanie o wymianę informacji do adresata \\
 107 & Odpowiedź na zapytanie o dołączenie \\
 108 & Odpowiedź na zapytanie o ruch \\
 109 & Odpowiedź na podniesienie kawałka \\
 110 & Odpowiedź na położenie kawałka \\
 111 & Wiadomość przekazująca odpowiedź na zapytanie o wymianę informacji \\
 \hline
\end{tabular}

\subsubsection{Wiadomości błędów}
\begin{tabular}{ |p{1cm}|p{12cm}| }
 \hline
 Id & Nazwa wiadomości \\
 \hline
 901 & Błędny ruch \\
 902 & Błędne odłożenie kawałka \\
 903 & Błędne położenie kawałka \\
 904 & Nie odczekanie kary \\
 905 & Niezdefiniowany błąd \\
 \hline
\end{tabular}

\end{document}
