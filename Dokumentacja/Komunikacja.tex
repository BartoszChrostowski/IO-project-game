\documentclass[Dokumentacja.tex]{subfiles}

\begin{document}
\section{Komunikacja}
Komunikacja odbywa się za pośrednictwem protokołu TCP. Wiadomości wysyłane są w formacie JSON.
Wiadomości wysyłane przez GM i Agenty opakowywane są w adapter opisujący kontekst wysyłanej wiadomości.
Pole \textit{agentID} uzupełniane jest przez serwer na podstawie. Po nawiązaniu połączenia przez Agenta z serwerem,
serwer nadaje mu unikane \textit{agentID} (liczbę całkowitą), które będzie służyło do korelacji oraz do będzie
wykorzystywane przez GM w logice gry. Te samo \textit{agentID} wysyłane jest do Agenta w odpowiedzi na prośbę o dołączenie
oraz w wiadomości o rozpoczęciu rozgrywki.
\lstinputlisting[language=json]{./DefinicjeWiadomosci/messageWrapper.json}

\subsection{Wiadomości Agenta}
\lstinputlisting[language=json]{./DefinicjeWiadomosci/AgentMes/checkForScham.json}
\lstinputlisting[language=json]{./DefinicjeWiadomosci/AgentMes/destroyPiece.json}
\lstinputlisting[language=json]{./DefinicjeWiadomosci/AgentMes/discovery.json}
\lstinputlisting[language=json]{./DefinicjeWiadomosci/AgentMes/infomationExchangeRespond.json}
\lstinputlisting[language=json]{./DefinicjeWiadomosci/AgentMes/informationExchangeAsk.json}
\lstinputlisting[language=json]{./DefinicjeWiadomosci/AgentMes/joinGame.json}
\lstinputlisting[language=json]{./DefinicjeWiadomosci/AgentMes/makeMove.json}
\lstinputlisting[language=json]{./DefinicjeWiadomosci/AgentMes/pickPiece.json}
\lstinputlisting[language=json]{./DefinicjeWiadomosci/AgentMes/putPiece.json}

\end{document}
