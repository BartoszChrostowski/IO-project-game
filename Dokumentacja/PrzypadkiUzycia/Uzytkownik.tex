\documentclass[../Dokumentacja.tex]{subfiles}
\begin{document}
\subsection{Użytkownik}
Użytkownik jest jest osobą obsługującą The Project Game.
Inicjuje wszystkie moduły. Posiada wymienione poniżej możliwości ingerencji
w działanie poszczególnych modułów.
\subsubsection{Moduł Serwera Komunikacyjnego}
\includeDiagram[width=\textwidth]{resources/USER-SERVER.pdf}{Diagram przypadków użycia modułu Serwera Komunikacyjnego i aktora Użytkownik}

\begin{itemize}
    \item Włączanie serwera
    \begin{itemize}
    	\item Ustawianie portów do połączeń z GM oraz agentami.
    	\item Ustawianie limitu ilości połączeń.
    \end{itemize}
    \item Wyłączanie serwera w dowolnym momencie
\end{itemize}

\subsubsection{Moduł agenta}
\includeDiagram[width=\textwidth]{resources/USER-AGENT.pdf}{Diagram przypadków użycia modułu Agent i aktora Użytkownik}
\begin{itemize}
	\item Inicjowanie agenta
	\begin{itemize}
		\item Zainicjowanie agenta wymaga parametrów połączenia z Serwerem Komunikacyjnym.
		\item Wyspecyfikowanie do której drużyny należy agent (Red, Blue).
		\item Wybranie strategii używanej podczas gry przez agenta spośród zaimplementowanych w module agenta. Od wybranej strategii zależy sposób podejmowania przez agenta takich decyzji jak: przesunięcie się na planszy, opuszczenie fragmentu, czy odpowiedź innemu agentowi na zapytanie. Wybór strategi sprowadza się do wybrania identyfikatora strategii spośród specyfikowanych przez konkretnego agenta.
	\end{itemize}
	\item Wyłączanie agenta
	\begin{itemize}
		\item Agenta można wyłączyć w każdym momencie, nie ma to wpływu na przebieg rozgrywki w której uczestniczył.
	\end{itemize}
\end{itemize}

\subsubsection{Moduł GM}
\includeDiagram[width=\textwidth]{resources/USER-GM.pdf}{Diagram przypadków użycia modułu GM i aktora Użytkownik}
\begin{itemize}
	\item Inicjowanie GM
	\begin{itemize}
		\item Zainicjowanie GM wymaga wprowadzanie parametrów połączeń.
	\end{itemize}
	\item Użytkownik ma możliwość skonfigurowania parametrów rozgrywki, przed jej rozpoczęciem
	\begin{itemize}
		\item Ilość celów w polu bramkowym
		\item Wymiary pola bramkowego
		\item Opóźnienie w wykonywaniu ruchów przez agenta
		\item Wymiary planszy
		\item Maksymalna ilość agentów
		\item Prawdopodobieństwo, że pojawiający się fragment jest fragmentem fikcyjnym
		\item Ilość fragmentów na planszy
	\end{itemize}
	\item Rozpoczynanie rozgrywki
	\begin{itemize}
		\item Użytkownik może zażądać rozpoczęcia rozgrywki, od tego momentu GM przechodzi w stan wyświetlania planszy i nie możliwa jest zmiana kofiguracji rozgrywki.
	\end{itemize}
	\item Wyświetlanie planszy
	\begin{itemize}
		\item Wyświetlenie graficznej interpretacji aktualnego stany gry. Na widoku znajduje się mapa z rozmieszeniem pól, Agentów, fragmetów i podstawowe statystyki rozgrywki.
	\end{itemize}
	\item Zakończenie rozgrywki
	\begin{itemize}
		\item W dowolnym momencie trwającej rozgrywki, użytkownik może zażądać zakończenia jej. Żądanie to przekierowuje na ekran z statystykami rozgrywki po czym kończy działanie GM.
	\end{itemize}
\end{itemize}
\end{document}
