\documentclass[./../Dokumentacja.tex]{subfiles}

\begin{document}
\section{Konfiguracje}
\label{sec:Konfiguracje}
\subsection{Serwer Komunikacyjny}
Konfiguracja Serwera Komunikacyjnego pobierana jest z pliku konfiguracyjnego w postaci JSON,
bądź jako parametry wywołania programu.
\lstinputlisting[language=json]{./resources/CS.jsonc}
\subsection{Agent}
Parametry pobierane są z pliku konfiguracyjnego w postaci JSON, w przypadku gdy taki nie istnieje
CLI odpytuje użytkownika o wprowadzenie potrzebnych parametrów.
\lstinputlisting[language=json]{./resources/Agent.jsonc}
\subsection{GM}
GM zgodnie z założeniami posiada GUI, parametry konfiguracyjne są wybierane przez użytkownika w
oknie ustawień przed rozpoczęciem rozgrywki, domyślne wartości uzupełniane są na podstawie pliku
konfiguracyjnego w postaci JSON.
\lstinputlisting[language=json]{./resources/GM.jsonc}
\subsection{Przykładowe pliki konfiguracyjne}
\subsubsection{Serwer Komunikacyjny}
\lstinputlisting[language=json]{./resources/CSConfig.json}
\subsubsection{Agent}
\lstinputlisting[language=json]{./resources/AgentConfig.json}
\subsubsection{GM}
\lstinputlisting[language=json]{./resources/GMConfig.json}
\end{document}
