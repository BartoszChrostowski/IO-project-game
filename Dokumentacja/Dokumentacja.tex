\documentclass[a4paper]{article}
\usepackage{subfiles}
\usepackage{blindtext} %\blindtext
\usepackage[T1]{fontenc}
\usepackage{lmodern}
\usepackage[polish]{babel}
\usepackage[utf8]{inputenc}
\selectlanguage{polish}
\usepackage[unicode]{hyperref}
\usepackage{amsfonts} %\mathbb
\usepackage{amsmath}   %\tag and \eqref
\usepackage{graphicx} %\includegraphics
\usepackage{outlines}
\usepackage{pdfpages}
\usepackage{geometry}
\usepackage{CustomCommands}
\usepackage{svg}
\usepackage{float}
\usepackage{hyperref}
\usepackage{pdflscape}

\title{IO - The Project Game}
\author{Bartosz Chrostowski
        \and Emil Dragańczuk
        \and Jakub Drak Sbahi
        \and Mikołaj Molenda
        \and Alicja Moskal}
\date{}

\begin{document}
\maketitle
\newpage
\tableofcontents
\newpage
\subfile{Jezyk.tex}
\newpage
\subfile{Reguly.tex}
\subfile{furps.tex}
\begin{samepage}
        \newpage
        \section{Diagram systemu}
        System złożony jest z 3 modułów komunikujących się ze sobą przez TCP.
        W celu uniknięcia skomplikowanej logiki buforowania wiadomości, każdy moduł kończy pracę na niepowodzenie komunikacji,
        System należy odpalać w kolejności CS -> GM -> Agenty, inna kolejność uruchamiania może powodować zamknięcie całego systemu kaskadowo.
        \includeDiagram[width=\textwidth]{resources/systemArchitecture.pdf}{Diagram systemu}
\end{samepage}

% Przypadki użycia
\subfile{./PrzypadkiUzycia/PrzypadkiUzycia.tex}
\subfile{./Komunikacja.tex}
\subfile{./Konfiguracje/Konfiguracje.tex}
\subfile{./DiagramyKlas/DiagramyKlas.tex}
\subfile{./DiagramyAktywnosci/DiagramyAktywnosci.tex}
\subfile{./DiagramySekwencji/DiagramySekwencji.tex}
\subfile{./DiagramyStanow/DiagramyStanow.tex}
\end{document}

