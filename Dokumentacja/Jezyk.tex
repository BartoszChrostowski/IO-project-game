\documentclass[Dokumentacja.tex]{subfiles}

\begin{document}
\paragraph{Język}
\begin{itemize}
    \item \textbf{GM} \textit{Game master} - moduł odpowiedzialny za prowadzenie rozgrywki
    \item \textbf{Agent} \textit{Player} - moduł sterujący graczem
    \item \textbf{Serwer komunikacyjny} \textit{Communications Server} - serwer odpowiedzialny za komunikację między Agentami a GM
    \item \textbf{plansza} - macierz na której rozgrywana jest gra
    \item \textbf{pole} \textit{field} - komórka macierzy planszy
    \item \textbf{fragment} \textit{piece} - obiekt pojawiający się na polu zadań, który należy położyć na cel
    \item \textbf{fragment fikcyjny} \textit{sham piece} - fragment który po położeniu w polu bramkowym nie powoduje informacji zwrotnej czy pole było celem
    \item \textbf{pole zadań} \textit{Tasks area} - zbiór pul niebędących polami bramkowymi na których mogą pojawić się fragmenty
    \item \textbf{cel} \textit{goal} - pole znajdujące się w polu bramkowym na które należy położyć fragment
    \item \textbf{pole bramkowe} \textit{Goals area} - zbiór pól na których mogą być cele
    \item \textbf{akcja} - czynność wykonywana przez agenta związana z rozgrywką
    \item \textbf{akcja odkrycia} \textit{discovery} - akcja agenta polegająca na odpytaniu o odległości przylegajacyh do agenta pól
\end{itemize}
\end{document}
