\documentclass[../Dokumentacja.tex]{subfiles}
\begin{document}
\subsection{Moduł Agenta}
\subsubsection{Agent}
\paragraph{Zmienne}
\begin{center}
    \begin{tabular}{ | l | l | l | p{5cm} |}
    \hline
    Dostęp & Typ & Nazwa & Opis \\ \hline
    private & int & id & Identyfikator gracza \\ \hline
    public & int & delayTime & Czas między wykonywanymi akcjami \\ \hline
    public & int & penaltyTime & Czas kary \\ \hline
    public & Team & team & Enum drużyny, w której jest gracz \\ \hline
    public & bool & isLeader & Zmienna informująca czy gracz jest liderem zespołu \\ \hline
    public & bool & havePiece & Zmienna informująca czy gracz jest w posiadaniu piece'a \\ \hline
    public & Field[,] & board & Tablica dwuwymiarowa przechowująca planszę z perspektywy gracza \\ \hline
    public & Tuple<int,int> & position & Współrzędne pola na planszy, na którym stoi gracz \\ \hline
    public & List<Tuple<int,bool>> & waitingPlayers & Lista id oraz
    informacji czy są oni liderami drużyny graczy oczekujących na odpowiedź \\ \hline
    public & int[] & teamMates & Lista identyfikatorów graczy z naszej drużyny \\ \hline
    private & IStrategy & strategy & Obiekt odpowiedzialny za używaną strategię \\ \hline
    \end{tabular}
\end{center}

\paragraph{Metody}
\begin{itemize}
    \method{Player()}{Konstruktor klasy}
    \method{JoinTheGame()}{Metoda dołączająca gracza do gry. Wysyła zapytanie do GM z prośbą o dołączenie.}
    \method{Start()}{Metoda rozpoczynająca grę. Rozpoczyna realizowanie strategii przez agenta.}
    \method{Stop()}{Metoda zatrzymująca grę. Zatrzymuję pracę agenta i wyłącza go.}
    \method{Move()}{Metoda ruszająca gracza na wskazane pole. Wysyła zapytanie do GM i w zależności od odpowiedzi aktualizuje stan wiedzy agneta.}
    \method{Put()}{Metoda odkładająca fragment na planszę, Wysyła zapytanie do GM i w zależności od odpowiedzi aktualizuje stan wiedzy agenta.}
    \method{BegForInfo()}{Metoda wysyłająca do GM prośbę o wymianę informacji przez innego wybranego agenta}
    \method{GiveInfo()}{Metoda udzielająca informacji o rozgrywce wskazanemu graczowi.}
    \method{RequestResponse()}{Metoda wywoływana na przyjście wiadomości o prośbie o wymianę informacji. Zapamiętuje parametry gracza który poprosił informację do listy \textit{waitingPlayers}}
    \method{CheckPiece()}{Metoda sprawdzająca czy fragment to fragment fikcyjny. Wysyła zapytanie do GM i w zależności od odpowiedzi aktualizuje stan wiedzy agenta.}
    \method[public void]{AcceptMessage()}{Metoda pobierająca informacje od communication serwera.}
    \method{MakeDecisionFromStrategy()}{Metoda, która wywołuje metodę z obiektu strategy. Wywoływana jest okresowo i to ona decuduje o akcjach wykonywanych przez agenta. Używa IStrategy}
    \method[private void]{Communicate()}{Metoda wysyłająca wiadomości do serwera komunikacyjnego. Używana przez metody wykonujące akcje do komunikacj z GM poprzez wysłanie wiadomości do Serwera Komunikacyjnego}
    \method[private void]{Penalty()}{Metoda czekająca przez okres kary. Po odczekaniu kary wywołuje generowanie kolejnej akcji.}
\end{itemize}


\subsubsection{Interface IStratety}
\paragraph{Metody}
\begin{itemize}
    \method{MakeDecision(Agent)}{Metoda decyduje na podstawie stanu Agenta jaką akcję wykonać i wykonuje ją na obiekcie Agenta.}
\end{itemize}
\subsubsection{Klasa Field}
\paragraph{Zmienne}
\begin{center}
    \begin{tabular}{ | l | l | l | p{5cm} |}
    \hline
    Dostęp & Typ & Nazwa & Opis \\ \hline
    public & GoalInfo & goalInfo & Stan wiedzy gracza na temat tego pola \\ \hline
    public & bool & playerInfo & Informacja czy inny gracz stoi na tym polu \\ \hline
    public & int & distToPiece & Odległość pola od najbliższego fragmentu \\ \hline
    \end{tabular}
\end{center}
\subsubsection{Enum Team}
\begin{itemize}
    \item \textbf{Red} - informacja, że gracz jest w drużynie czerwonej,
    \item \textbf{Blue} - informacja, że gracz jest w drużynie niebieskiej.
\end{itemize}
\subsubsection{Enum GoalInfo}
\begin{itemize}
    \item \textbf{IDK} - gracz nic nie wie o danym polu,
    \item \textbf{DiscoveredNotGoal} - pole nie jest celem,
    \item \textbf{DiscoveredGoal} - pole jest celem i jest na nim położony fragment,
\end{itemize}
\end{document}
