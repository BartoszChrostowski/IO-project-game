\documentclass[../Dokumentacja.tex]{subfiles}
\begin{document}
\subsection{Moduł Agenta}
\includeDiagram[width=\textwidth]{resources/Agent.eps}{Diagram klas modułu Agenta}
\subsubsection{Agent}
\begin{methods}
    \method[private int]{id}{Identyfikator gracza.}
    \method[public int]{penaltyTime}{Czas kary.}
    \method[public Team]{team}{Enum drużyny, w której jest gracz.}
    \method[public bool]{isLeader}{Zmienna informująca czy gracz jest liderem zespołu.}
    \method[public bool]{hasPiece}{Zmienna informująca czy gracz jest w posiadaniu piece'a.}
    \method[{public Field[,]}]{board}{Tablica dwuwymiarowa przechowująca planszę z perspektywy gracza.}
    \method[{public Tuple<int y,int x>}]{position}{Współrzędne pola na planszy, na którym stoi gracz.}
    \method[{public List<int>}]{waitingPlayers}{Lista id graczy oczekujących na odpowiedź.}
    \method[{public int[]}]{teamMates}{Lista identyfikatorów graczy z naszej drużyny.}
    \method[private IStrategy]{strategy}{Obiekt odpowiedzialny za używaną strategię.}
\end{methods}
\paragraph{Metody}
\begin{methods}
    \method{Player()}{Konstruktor klasy.}
    \method{JoinTheGame()}{Metoda dołączająca gracza do gry. Wysyła zapytanie do GM z prośbą o dołączenie.}
    \method{Start()}{Metoda rozpoczynająca grę. Rozpoczyna realizowanie strategii przez agenta.}
    \method{Stop()}{Metoda zatrzymująca grę. Zatrzymuje pracę agenta i wyłącza go.}
    \method{Move()}{Metoda ruszająca gracza na wskazane pole. Wysyła zapytanie do GM i w zależności od odpowiedzi aktualizuje stan wiedzy agneta.}
    \method{Put()}{Metoda odkładająca fragment na planszę. Wysyła zapytanie do GM i w zależności od odpowiedzi aktualizuje stan wiedzy agenta.}
    \method{BegForInfo()}{Metoda wysyłająca do GM prośbę o wymianę informacji przez innego wybranego agenta}
    \method{GiveInfo()}{Metoda udzielająca informacji o rozgrywce wskazanemu graczowi.}
    \method{RequestResponse()}{Metoda wywoływana na przyjście wiadomości z prośbą o wymianę informacji. Zapamiętuje parametry gracza który poprosił informację do listy \textit{waitingPlayers}. W przypadku gdy ten gracz był liderem wywołuje metodę \textit{GiveInfo()}}
    \method{CheckPiece()}{Metoda sprawdzająca czy fragment to fragment fikcyjny. Wysyła zapytanie do GM i w zależności od odpowiedzi aktualizuje stan wiedzy agenta.}
    \method[public void]{AcceptMessage()}{Metoda pobierająca informacje od serwera komunikacyjnego.}
    \method{MakeDecisionFromStrategy()}{Metoda, która wywołuje metodę z obiektu strategy. Wywoływana jest okresowo i to ona decyduje o akcjach wykonywanych przez agenta. Używa IStrategy.}
    \method[private void]{Communicate()}{Metoda wysyłająca wiadomości do serwera komunikacyjnego. Używana przez metody wykonujące akcje do komunikacj z GM poprzez wysłanie wiadomości do Serwera Komunikacyjnego}
    \method[private void]{Penalty()}{Metoda czekająca przez okres kary. Po odczekaniu kary wywołuje generowanie kolejnej akcji.}
\end{methods}


\subsubsection{Interface IStrategy}
\paragraph{Metody}
\begin{methods}
    \method{MakeDecision(Agent)}{Metoda decyduje na podstawie stanu Agenta jaką akcję wykonać i wykonuje ją na obiekcie Agenta.}
\end{methods}
\subsubsection{Klasa Field}
\begin{methods}
    \method[public GoalInfo]{goalInfo}{Stan wiedzy gracza na temat tego pola.}
    \method[public bool]{playerInfo}{Informacja czy inny gracz stoi na tym polu.}
    \method[public int]{distToPiece}{Odległość pola od najbliższego fragmentu.}
\end{methods}
\subsubsection{Enum Team}
\begin{itemize}
    \item \textbf{Red} - informacja, że gracz jest w drużynie czerwonej,
    \item \textbf{Blue} - informacja, że gracz jest w drużynie niebieskiej.
\end{itemize}
\subsubsection{Enum GoalInfo}
\begin{itemize}
    \item \textbf{IDK} - gracz nic nie wie o danym polu,
    \item \textbf{DiscoveredNotGoal} - pole nie jest celem,
    \item \textbf{DiscoveredGoal} - pole jest celem i jest na nim położony fragment.
\end{itemize}
\end{document}
