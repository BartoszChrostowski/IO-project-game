\documentclass[../Dokumentacja.tex]{subfiles}
\begin{document}

\subsection{Moduł Serwera Komunikacyjnego}
\includeDiagram[ ]{resources/CS.eps}{Diagram klas modułu Serwera Komunikacyjnego}

\subsubsection{Klasa Communicator}
\paragraph{Zmienne}
\begin{methods}
    \method[private Dictionary<int, Descriptor>]{correlation}{Słownik mapujący wewnętrzne id korelacji gry na deskryptory do wysłania wiadomości wykorzystywane przez API do komunikacji}
    \method[private Descriptor]{GMDescriptor}{Deskryptor używany do wysyłania wiadomości do GM}
    \method[private ISender]{senderService}{Obiekt wykorzystywany do wysyłania wiadomości}
\end{methods}
\paragraph{Metody}
\begin{methods}
    \method{SendMessageToAgent(GMMessage)}{Metoda wysyłająca odpowiednią wiadomość wiadomość do Agenta. Na podstawie id z wiadomości znajduje odpowiedni deskryptor i wysyła na niego payload za pomocą ISender. Wywoływana przez API do komunikacji na zdarzenie otrzymania wiadomości od Agenta}
    \method{SendMessageToGM(AgentMessage)}{Metoda wysyłająca odpowiednią wiadomość wiadomość do Game Mastera. Wywoływana przez API do komunikacji na zdarzenie otrzymania wiadomości od GM}
\end{methods}

\subsubsection{Klasa GMMessage}
\begin{methods}
    \method[public int]{id}{Identyfikator agenta do którego przesyłana jest wiadomość.}
	\method[public string]{payload}{Treść wiadomości przesyłana do agenta w formacie JSON.}
\end{methods}

\subsubsection{Klasa AgentMessage}
\begin{methods}
	\method[public string]{payload}{Treść wiadomości przesyłana do Game Mastera w formacie JSON.}
\end{methods}

\subsubsection{Interfejs ISender}
\begin{methods}
    \method[public void]{SendMessage()}{Metoda odpowiedzialna za wysłanie za pomocą API do komunikacji wiadomości do odpowiedniego Agenta lub GM w zależności od paramterów wywołania }
\end{methods}

\end{document}
