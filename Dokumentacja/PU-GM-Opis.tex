\documentclass[a4paper]{article}
%\usepackage{blindtext} %\blindtext
\usepackage[T1]{fontenc}
\usepackage{lmodern}
\usepackage[polish]{babel} 
\usepackage[utf8]{inputenc}
\usepackage{url}
\selectlanguage{polish}
\usepackage{amsfonts} %\mathbb
\usepackage{amsmath}   %\tag and \eqref 
\usepackage{graphicx} %\includegraphics
\usepackage{biblatex}
\graphicspath{ {./graphics/} }
\title{Przypadki użycia - Game Master}
\date{Listopadk\\2019}
\author{Emil Dragańczuk}
\begin{document}
\maketitle
\section{}
\section{Przypadki użycia}
\subsection{Game Master}
\begin{itemize}
	\item Rozpoczęcie rozgrywki
	\begin{itemize}
		\item Pobranie parametrów rozgrywki z konfiguracji
		\item Wygenerowanie nowej planszy
		\item Wysłanie informacji o rozpoczęciu do wszystkich agentów
	\end{itemize}
	\item Obsługa prosby o dołączenie
	\begin{itemize}
		\item Game Master może odrzucić lub zaakceptować prosbę, w zależnosci od limitu agentow w drużynie
	\end{itemize}
	\item Obsługa zapytań Agentów
	\begin{itemize}
		\item Game Master na podstawie globalnego stanu planszy waliduje zapytanie Agenta i odsyła spowrotem wynik walidacji
		\item Jesli zapytanie dotyczy stanu planszy to Game Master odsyła odpowiedź na podstawie aktualnego stanu planszy
		\item Potencjalnie akcja agenta może spowodować zmianę stanu planszy, w tym wygenerowanie nowego kawałka jesli akcją było dostarczenie kawałka do celu
	\end{itemize}
	\item Prezentowanie stanu planszy wraz z wynikiem
	\begin{itemize}
		\item Game Master na swoim interfejsie użytkownika prezentuje aktualny stan planszy, wynik oraz meczowe statystyki
	\end{itemize}
	\item Zakończenie rozgrywki
	\begin{itemize}
		\item Na żądanie użytkownika lub po wygranej którejkolwiek z drużyn Game Master wysyła informacje o końcu gry do Agentów po czym sam się wyłącza
	\end{itemize}
\end{itemize}

\end{document}

