\documentclass[../Dokumentacja.tex]{subfiles}

\begin{document}
\section{Diagramy Stanów}
\subsection{GM}
\includeDiagram[width=\textwidth]{resources/GM.eps}{Diagram stanów Game Mastera}
\subsection{Agent}
\includeDiagramFullPage[width=\textwidth]{resources/AgentStates.pdf}{Diagram stanów Agenta}
\subsection{Kawałek}
Stan kawałka może zmieniać jedynie Agent akcjami przez niego wykonywalnymi. Kawałki są usuwane (\textit{jako obiekty}) tylko na
zakończenie rozgrywki oraz na akcję zniszczenia. Odłożony kawałek nie zawsze może zostać podniesiony, o tym czy dany kawałek może
zostać podniesiony decyduje typ pola na który został odłożony.
\includeDiagram[width=\textwidth]{resources/Piece.pdf}{Diagram stanów kawałka}
\subsection{Pole bramkowe}
Pole bramkowe rozpoczyna jako puste. Zgodnie z regułami raz odłożony fragment na pole bramkowe nie może zostać już podniesiony, zatem
po położeniu na pole bramkowe fragmentu te już nigdy nie przejdzie do stanu pustego. GM nie może wygenerować fragmentu na polu bramkowym
\includeDiagramFullPage[width=\textwidth]{resources/Goal.pdf}{Diagram stanów pola w polu bramkwym}
\end{document}
